\documentclass[review]{elsarticle}

\usepackage{amsmath,amssymb,amsthm,xspace,mathtools,hyperref,color}

\newcommand{\bvec}[1]{\boldsymbol{#1}}
%\newcommand{\bvec}[1]{\text{\boldmath$#1$}}
\newcommand{\avec}[1]{\vec{#1}}
%\renewcommand{\vec}[1] {\text{\boldmath$#1$}}
%\renewcommand{\vec}[1]{\ensuremath{\mathbf{#1}}}
\newcommand{\vecsym}[1]{\ensuremath{\boldsymbol{#1}}}
\def\bbl{\text{\boldmath$\{$}}
\def\bbr{\text{\boldmath$\}$}}
\newcommand{\bbrace}[1]{\bbl #1 \bbr}
\newcommand{\bbbrace}[1]{\mathopen{\pmb{\bigg\{}}#1\mathclose{\pmb{\bigg\}}}}
%\def\bbl{\boldsymbol{\left \{}}
%\def\bbr{\boldsymbol{\right \}}}
\def\betahat{\hat\beta}
%\def\e{\text{e}}
%\def\E{\text{E}}
\newcommand{\dif}{{\rm d}}

\newlength{\overwdth}
\def\overstrike#1{ 
\settowidth{\overwdth}{#1}\makebox[0pt][l]{\rule[0.5ex]{\overwdth}{0.1ex}}#1}

\def\abs#1{\ensuremath{\left \lvert #1 \right \rvert}}
\newcommand{\bigabs}[1]{\ensuremath{\bigl \lvert #1 \bigr \rvert}}
\newcommand{\Bigabs}[1]{\ensuremath{\Bigl \lvert #1 \Bigr \rvert}}
\newcommand{\biggabs}[1]{\ensuremath{\biggl \lvert #1 \biggr \rvert}}
\newcommand{\Biggabs}[1]{\ensuremath{\Biggl \lvert #1 \Biggr \rvert}}
\newcommand{\norm}[2][{}]{\ensuremath{\left \lVert #2 \right \rVert}_{#1}}
\newcommand{\bignorm}[2][{}]{\ensuremath{\bigl \lVert #2 \bigr \rVert}_{#1}}
\newcommand{\Bignorm}[2][{}]{\ensuremath{\Bigl \lVert #2 \Bigr \rVert}_{#1}}
\newcommand{\biggnorm}[2][{}]{\ensuremath{\biggl \lVert #2 \biggr \rVert}_{#1}}
\newcommand{\ip}[3][{}]{\ensuremath{\left \langle #2, #3 \right \rangle_{#1}}}

\newcommand{\bigvecpar}[3]{\ensuremath{\bigl ( #1 \bigr )_{#2}^{#3}}}
\newcommand{\Bigvecpar}[3]{\ensuremath{\Bigl ( #1 \Bigr )_{#2}^{#3}}}
\newcommand{\biggvecpar}[3]{\ensuremath{\biggl ( #1 \biggr )_{#2}^{#3}}}
\newcommand{\bigpar}[1]{\ensuremath{\bigl ( #1 \bigr )}}
\newcommand{\Bigpar}[1]{\ensuremath{\Bigl ( #1 \Bigr )}}
\newcommand{\biggpar}[1]{\ensuremath{\biggl ( #1 \biggr )}}

\newcommand{\IIDsim}{\overset{\textup{IID}}{\sim}}

\DeclareMathOperator{\success}{succ}
\DeclareMathOperator{\sinc}{sinc}
\DeclareMathOperator{\sech}{sech}
\DeclareMathOperator{\csch}{csch}
\DeclareMathOperator{\dist}{dist}
\DeclareMathOperator{\spn}{span}
\DeclareMathOperator{\sgn}{sgn}
\DeclareMathOperator*{\rmse}{rmse}
\DeclareMathOperator{\Prob}{\mathbb{P}}
\DeclareMathOperator{\Ex}{\mathbb{E}}
\DeclareMathOperator{\rank}{rank}
\DeclareMathOperator{\erfc}{erfc}
\DeclareMathOperator{\erf}{erf}
\DeclareMathOperator{\cov}{cov}
\DeclareMathOperator{\cost}{cost}
\DeclareMathOperator{\comp}{comp}
\DeclareMathOperator{\corr}{corr}
\DeclareMathOperator{\diag}{diag}
\DeclareMathOperator{\var}{var}
\DeclareMathOperator{\opt}{opt}
\DeclareMathOperator{\brandnew}{new}
\DeclareMathOperator{\std}{std}
\DeclareMathOperator{\kurt}{kurt}
\DeclareMathOperator{\med}{med}
\DeclareMathOperator{\vol}{vol}
\DeclareMathOperator{\bias}{bias}
\DeclareMathOperator*{\argmin}{argmin}
\DeclareMathOperator{\sign}{sign}
\DeclareMathOperator{\spann}{span}
\DeclareMathOperator{\cond}{cond}
\DeclareMathOperator{\trace}{trace}
\DeclareMathOperator{\Si}{Si}
%\DeclareMathOperator{\diag}{diag}
\DeclareMathOperator{\col}{col}
\DeclareMathOperator{\nullspace}{null}
\DeclareMathOperator{\Order}{{\mathcal O}}
%\DeclareMathOperator{\rank}{rank}

\newcommand{\vzero}{\bvec{0}}
\newcommand{\vone}{\bvec{1}}
\newcommand{\vinf}{\bvec{\infty}}
\newcommand{\va}{\bvec{a}}
\newcommand{\vA}{\bvec{A}}
\newcommand{\vb}{\bvec{b}}
\newcommand{\vB}{\bvec{B}}
\newcommand{\vc}{\bvec{c}}
\newcommand{\vd}{\bvec{d}}
\newcommand{\vD}{\bvec{D}}
\newcommand{\ve}{\bvec{e}}
\newcommand{\vf}{\bvec{f}}
\newcommand{\vF}{\bvec{F}}
\newcommand{\vg}{\bvec{g}}
\newcommand{\vG}{\bvec{G}}
\newcommand{\vh}{\bvec{h}}
\newcommand{\vi}{\bvec{i}}
\newcommand{\vj}{\bvec{j}}
\newcommand{\vk}{\bvec{k}}
\newcommand{\vl}{\bvec{l}}
\newcommand{\vell}{\bvec{\ell}}
\newcommand{\vL}{\bvec{L}}
\newcommand{\vm}{\bvec{m}}
\newcommand{\vp}{\bvec{p}}
\newcommand{\vq}{\bvec{q}}
\newcommand{\vr}{\bvec{r}}
\newcommand{\vs}{\bvec{s}}
\newcommand{\vS}{\bvec{S}}
\newcommand{\vt}{\bvec{t}}
\newcommand{\vT}{\bvec{T}}
\newcommand{\vu}{\bvec{u}}
\newcommand{\vU}{\bvec{U}}
\newcommand{\vv}{\bvec{v}}
\newcommand{\vV}{\bvec{V}}
\newcommand{\vw}{\bvec{w}}
\newcommand{\vW}{\bvec{W}}
\newcommand{\vx}{\bvec{x}}
\newcommand{\vX}{\bvec{X}}
\newcommand{\vy}{\bvec{y}}
\newcommand{\vY}{\bvec{Y}}
\newcommand{\vz}{\bvec{z}}
\newcommand{\vZ}{\bvec{Z}}

\newcommand{\ai}{\avec{\imath}}
\newcommand{\ak}{\avec{k}}
\newcommand{\avi}{\avec{\bvec{\imath}}}
\newcommand{\at}{\avec{t}}
\newcommand{\avt}{\avec{\vt}}
\newcommand{\ax}{\avec{x}}
\newcommand{\ah}{\avec{h}}
\newcommand{\akappa}{\avec{\kappa}}
\newcommand{\avx}{\avec{\vx}}
\newcommand{\ay}{\avec{y}}
\newcommand{\avy}{\avec{\vy}}
\newcommand{\avz}{\avec{\vz}}
\newcommand{\avzero}{\avec{\vzero}}
\newcommand{\aomega}{\avec{\omega}}
\newcommand{\avomega}{\avec{\vomega}}
\newcommand{\anu}{\avec{\nu}}
\newcommand{\avnu}{\avec{\vnu}}
\newcommand{\aDelta}{\avec{\Delta}}
\newcommand{\avDelta}{\avec{\vDelta}}

\newcommand{\valpha}{\bvec{\alpha}}
\newcommand{\vbeta}{\bvec{\beta}}
\newcommand{\vgamma}{\bvec{\gamma}}
\newcommand{\vdelta}{\bvec{\delta}}
\newcommand{\vDelta}{\bvec{\Delta}}
\newcommand{\vphi}{\bvec{\phi}}
\newcommand{\vvphi}{\bvec{\varphi}}
\newcommand{\vomega}{\bvec{\omega}}
\newcommand{\vlambda}{\bvec{\lambda}}
\newcommand{\vmu}{\bvec{\mu}}
\newcommand{\vnu}{\bvec{\nu}}
\newcommand{\vpsi}{\bvec{\psi}}
\newcommand{\vepsilon}{\bvec{\epsilon}}
\newcommand{\veps}{\bvec{\varepsilon}}
\newcommand{\veta}{\bvec{\eta}}
\newcommand{\vxi}{\bvec{\xi}}
\newcommand{\vtheta}{\bvec{\theta}}
\newcommand{\vtau}{\bvec{\tau}}
\newcommand{\vzeta}{\bvec{\zeta}}

\newcommand{\hvb}{\hat{\vb}}
\newcommand{\hcc}{\widehat{\cc}}
\newcommand{\hD}{\widehat{D}}
\newcommand{\hE}{\widehat{E}}
\newcommand{\hf}{\widehat{f}}
\newcommand{\hF}{\widehat{F}}
\newcommand{\hg}{\hat{g}}
\newcommand{\hvf}{\widehat{\bvec{f}}}
\newcommand{\hh}{\hat{h}}
\newcommand{\hH}{\widehat{H}}
\newcommand{\hi}{\hat{\imath}}
\newcommand{\hI}{\hat{I}}
\newcommand{\hci}{\widehat{\ci}}
\newcommand{\hj}{\hat{\jmath}}
\newcommand{\hp}{\hat{p}}
\newcommand{\hP}{\hat{P}}
\newcommand{\hS}{\widehat{S}}
\newcommand{\hv}{\hat{v}}
\newcommand{\hV}{\widehat{V}}
\newcommand{\hx}{\hat{x}}
\newcommand{\hX}{\widehat{X}}
\newcommand{\hvX}{\widehat{\vX}}
\newcommand{\hy}{\hat{y}}
\newcommand{\hvy}{\hat{\vy}}
\newcommand{\hY}{\widehat{Y}}
\newcommand{\hvY}{\widehat{\vY}}
\newcommand{\hZ}{\widehat{Z}}
\newcommand{\hvZ}{\widehat{\vZ}}

\newcommand{\halpha}{\hat{\alpha}}
\newcommand{\hvalpha}{\hat{\valpha}}
\newcommand{\hbeta}{\hat{\beta}}
\newcommand{\hvbeta}{\hat{\vbeta}}
\newcommand{\hgamma}{\hat{\gamma}}
\newcommand{\hvgamma}{\hat{\vgamma}}
\newcommand{\hdelta}{\hat{\delta}}
\newcommand{\hvareps}{\hat{\varepsilon}}
\newcommand{\hveps}{\hat{\veps}}
\newcommand{\hmu}{\hat{\mu}}
\newcommand{\hnu}{\hat{\nu}}
\newcommand{\hvnu}{\widehat{\vnu}}
\newcommand{\homega}{\widehat{\omega}}
\newcommand{\hrho}{\hat{\rho}}
\newcommand{\hsigma}{\hat{\sigma}}
\newcommand{\htheta}{\hat{\theta}}
\newcommand{\hTheta}{\hat{\Theta}}
\newcommand{\htau}{\hat{\tau}}
\newcommand{\hxi}{\hat{\xi}}
\newcommand{\hvxi}{\hat{\vxi}}

\newcommand{\otau}{\overline{\tau}}
\newcommand{\oY}{\overline{Y}}

\newcommand{\rD}{\mathring{D}}
\newcommand{\rf}{\mathring{f}}
\newcommand{\rV}{\mathring{V}}

\newcommand{\ta}{\tilde{a}}
\newcommand{\tA}{\tilde{A}}
\newcommand{\tmA}{\widetilde{\mA}}
\newcommand{\tvb}{\tilde{\vb}}
\newcommand{\tcb}{\widetilde{\cb}}
\newcommand{\tB}{\widetilde{B}}
\newcommand{\tc}{\tilde{c}}
\newcommand{\tvc}{\tilde{\vc}}
\newcommand{\tfc}{\tilde{\fc}}
\newcommand{\tC}{\widetilde{C}}
\newcommand{\tcc}{\widetilde{\cc}}
\newcommand{\tD}{\widetilde{D}}
\newcommand{\te}{\tilde{e}}
\newcommand{\tE}{\widetilde{E}}
\newcommand{\tf}{\widetilde{f}}
\newcommand{\tF}{\widetilde{F}}
\newcommand{\tvf}{\tilde{\vf}}
\newcommand{\tcf}{\widetilde{\cf}}
\newcommand{\tg}{\tilde{g}}
\newcommand{\tG}{\widetilde{G}}
\newcommand{\tildeh}{\tilde{h}}
\newcommand{\tH}{\widetilde{H}}
\newcommand{\tch}{\widetilde{\ch}}
\newcommand{\tK}{\widetilde{K}}
\newcommand{\tvk}{\tilde{\vk}}
\newcommand{\tM}{\widetilde{M}}
\newcommand{\tn}{\tilde{n}}
\newcommand{\tN}{\widetilde{N}}
\newcommand{\tQ}{\widetilde{Q}}
\newcommand{\tR}{\widetilde{R}}
\newcommand{\tS}{\widetilde{S}}
\newcommand{\tvS}{\widetilde{\vS}}
\newcommand{\tT}{\widetilde{T}}
\newcommand{\tv}{\tilde{v}}
\newcommand{\tV}{\widetilde{V}}
\newcommand{\tvx}{\tilde{\vx}}
\newcommand{\tW}{\widetilde{W}}
\newcommand{\tx}{\tilde{x}}
\newcommand{\tX}{\widetilde{X}}
\newcommand{\tvX}{\widetilde{\vX}}
\newcommand{\ty}{\tilde{y}}
\newcommand{\tvy}{\tilde{\vy}}
\newcommand{\tz}{\tilde{z}}
\newcommand{\tZ}{\widetilde{Z}}
\newcommand{\tL}{\tilde{L}}
\newcommand{\tP}{\widetilde{P}}
\newcommand{\tY}{\widetilde{Y}}
\newcommand{\tmH}{\widetilde{\mH}}
\newcommand{\tmK}{\widetilde{\mK}}
\newcommand{\tmM}{\widetilde{\mM}}
\newcommand{\tmQ}{\widetilde{\mQ}}
\newcommand{\tct}{\widetilde{\ct}}
\newcommand{\talpha}{\tilde{\alpha}}
\newcommand{\tdelta}{\tilde{\delta}}
\newcommand{\tDelta}{\tilde{\Delta}}
\newcommand{\tvareps}{\tilde{\varepsilon}}
\newcommand{\tveps}{\tilde{\veps}}
\newcommand{\tlambda}{\tilde{\lambda}}
\newcommand{\tmu}{\tilde{\mu}}
\newcommand{\tnu}{\tilde{\nu}}
\newcommand{\trho}{\tilde{\rho}}
\newcommand{\tvarrho}{\tilde{\varrho}}
\newcommand{\ttheta}{\tilde{\theta}}
\newcommand{\tsigma}{\tilde{\sigma}}
\newcommand{\tvmu}{\tilde{\vmu}}
\newcommand{\tphi}{\tilde{\phi}}
\newcommand{\tPhi}{\widetilde{\Phi}}
\newcommand{\tvphi}{\tilde{\vphi}}
\newcommand{\ttau}{\tilde{\tau}}
\newcommand{\txi}{\tilde{\xi}}
\newcommand{\tvxi}{\tilde{\vxi}}


\newcommand{\mA}{\mathsf{A}}
\newcommand{\mB}{\mathsf{B}}
\newcommand{\mC}{\mathsf{C}}
\newcommand{\vmC}{\bvec{\mC}}
\newcommand{\mD}{\mathsf{D}}
\newcommand{\mF}{\mathsf{F}}
\newcommand{\mG}{\mathsf{G}}
\newcommand{\mH}{\mathsf{H}}
\newcommand{\mI}{\mathsf{I}}
\newcommand{\mK}{\mathsf{K}}
\newcommand{\mL}{\mathsf{L}}
\newcommand{\mM}{\mathsf{M}}
\newcommand{\mP}{\mathsf{P}}
\newcommand{\mQ}{\mathsf{Q}}
\newcommand{\mR}{\mathsf{R}}
\newcommand{\mS}{\mathsf{S}}
\newcommand{\mT}{\mathsf{T}}
\newcommand{\mU}{\mathsf{U}}
\newcommand{\mV}{\mathsf{V}}
\newcommand{\mW}{\mathsf{W}}
\newcommand{\mX}{\mathsf{X}}
\newcommand{\mLambda}{\mathsf{\Lambda}}
\newcommand{\mSigma}{\mathsf{\Sigma}}
\newcommand{\mzero}{\mathsf{0}}
\newcommand{\mGamma}{\mathsf{\Gamma}}

\newcommand{\bbE}{\mathbb{E}}
\newcommand{\bbF}{\mathbb{F}}
\newcommand{\bbK}{\mathbb{K}}
\newcommand{\bbZ}{\mathbb{Z}}
\newcommand{\bbone}{\mathbbm{1}}
\newcommand{\naturals}{\mathbb{N}}
\newcommand{\reals}{\mathbb{R}}
\newcommand{\integers}{\mathbb{Z}}
\newcommand{\natzero}{\mathbb{N}_{0}}
\newcommand{\rationals}{\mathbb{Q}}
\newcommand{\complex}{\mathbb{C}}

\newcommand{\ca}{\mathcal{A}}
\newcommand{\cb}{\mathcal{B}}
\providecommand{\cc}{\mathcal{C}}
\newcommand{\cd}{\mathcal{D}}
\newcommand{\cf}{\mathcal{F}}
\newcommand{\cg}{\mathcal{G}}
\newcommand{\ch}{\mathcal{H}}
\newcommand{\ci}{\mathcal{I}}
\newcommand{\cj}{\mathcal{J}}
\newcommand{\ck}{\mathcal{K}}
\newcommand{\cl}{\mathcal{L}}
\newcommand{\cm}{\mathcal{M}}
\newcommand{\tcm}{\widetilde{\cm}}
\newcommand{\cn}{\mathcal{N}}
\newcommand{\cp}{\mathcal{P}}
\newcommand{\calr}{\mathcal{R}}
\newcommand{\cs}{\mathcal{S}}
\newcommand{\ct}{\mathcal{T}}
\newcommand{\cu}{\mathcal{U}}
\newcommand{\cv}{\mathcal{V}}
\newcommand{\cw}{\mathcal{W}}
\newcommand{\cx}{\mathcal{X}}
\newcommand{\tcx}{\widetilde{\cx}}
\newcommand{\cy}{\mathcal{Y}}
\newcommand{\cz}{\mathcal{Z}}

\newcommand{\fc}{\mathfrak{c}}
\newcommand{\fC}{\mathfrak{C}}
\newcommand{\fh}{\mathfrak{h}}
\newcommand{\fu}{\mathfrak{u}}

\newcommand{\me}{\ensuremath{\mathrm{e}}} % for math number 'e', 2.718 281 8..., tha base of natural logarithms
\newcommand{\mi}{\ensuremath{\mathrm{i}}} % for math number 'i', the imaginary unit
\newcommand{\mpi}{\ensuremath{\mathrm{\pi}}} % for math number 'pi', the circumference of a circle of diameter 1


\journal{Journal of Complexity}

%%%%%%%%%%%%%%%%%%%%%%%
%% Elsevier bibliography styles
%%%%%%%%%%%%%%%%%%%%%%%
%% To change the style, put a % in front of the second line of the current style and
%% remove the % from the second line of the style you would like to use.
%%%%%%%%%%%%%%%%%%%%%%%

%% Numbered
%\bibliographystyle{model1-num-names}

%% Numbered without titles
%\bibliographystyle{model1a-num-names}

%% Harvard
%\bibliographystyle{model2-names.bst}\biboptions{authoryear}

%% Vancouver numbered
%\usepackage{numcompress}\bibliographystyle{model3-num-names}

%% Vancouver name/year
\usepackage{numcompress}\bibliographystyle{model4-names}\biboptions{authoryear}

%% APA style
%\bibliographystyle{model5-names}\biboptions{authoryear}

%% AMA style
%\usepackage{numcompress}\bibliographystyle{model6-num-names}

%% `Elsevier LaTeX' style
%\bibliographystyle{elsarticle-num}
%%%%%%%%%%%%%%%%%%%%%%%

\newcommand{\abstol}{\varepsilon_{\textrm{a}}}
\newcommand{\datasites}{\{x_i\}_{i=0}^n}
\makeatletter
\newcommand{\vast}{\bBigg@{4}}
\newcommand{\Vast}{\bBigg@{6}}
\makeatother
%\newdefinition{algo}{Algorithm}
\theoremstyle{definition}
\newtheorem*{algoA}{Algorithm $A$}
\newcommand{\vastl}{\mathopen\vast}
\newcommand{\vastm}{\mathrel\vast}
\newcommand{\vastr}{\mathclose\vast}
\newcommand{\Vastl}{\mathopen\Vast}
\newcommand{\Vastm}{\mathrel\Vast}
\newcommand{\Vastr}{\mathclose\Vast}

\definecolor{orange}{rgb}{1.0,0.3,0.0}
\definecolor{violet}{rgb}{0.75,0,1}
\definecolor{darkgreen}{rgb}{0,0.6,0}
\newcommand{\frednote}[1]{  {\textcolor{red}  {\mbox{**Fred:} #1}}}
\newcommand{\yuhannote}[1]{  {\textcolor{darkgreen}  {\mbox{**Yuhan:} #1}}}
\newcommand{\xinnote}[1]{ {\textcolor{violet}  {\mbox{**Xin:} #1}}}
\newcommand{\scnote}[1]{ {\textcolor{orange}  {\mbox{**SC:} #1}}}

\newcommand{\Ixl}{I_{x,l}}
\newcommand{\Ixlx}{I_{x,\ell(x)}}
\newcommand{\Ixrlx}{I_{x,\rell(x)}}
\newcommand{\Ixhlx}{I_{x,\hell(x)}}
\newcommand{\hell}{\hat{\ell}}
\newcommand{\tell}{\tilde{\ell}}
\newcommand{\rell}{\mathring{\ell}}
\DeclareMathOperator{\lo}{lo}
\DeclareMathOperator{\ninit}{ninit}
\DeclareMathOperator{\errest}{errest}
\newtheorem{theorem}{Theorem}

\begin{document}

\begin{frontmatter}

\title{Local Adpation for Approximation and Optimization of Univariate Functions}


%% Group authors per affiliation:
\author{Sou-Cheng Choi}
\author{Yuhan Ding}
\author{Fred J. Hickernell}
\author{Xin Tong}
\address{Department of Applied Mathematics, Illinois Institute of Technology, RE 208, 10 West 32$^{\text{nd}}$ Street, Chicago, Illinois, 60616, USA}

\begin{abstract}
Some ideas to get us going
\end{abstract}

\begin{keyword}
\sep \sep
\MSC[2010]  \sep
\end{keyword}

\end{frontmatter}

%%%%%%%%%%%%%%%%%%%%%%%%%%%%%%%%%%%%%%%%%%%%%%%%%%
\section{Introduction} \label{sec:intro}
%%%%%%%%%%%%%%%%%%%%%%%%%%%%%%%%%%%%%%%%%%%%%%%%%%

The goal is to solve two problems by locally adaptive algorithms. For some suitable set, $\cc$, of continuous,
real-valued functions on the finite interval $[a,b]$, we  construct
algorithms $A:(\cc,(0,\infty)) \to L^{\infty}[a,b]$ and $M: (\cc,(0,\infty)) \to
\reals$ such that for any $f \in \cc$ and any error tolerance $\abstol > 0$,
\begin{gather}
\norm[\infty]{f - A(f,\abstol)} \le \abstol,  \tag{APP} \label{appxprob} \\
M(f,\abstol) - \min_{a \le x \le b} f(x)  \le \abstol. \tag{MIN} \label{optprob}
\end{gather}
Here $A$ and $M$ depend only on function values. They choose the data sites in $[a,b]$ adaptively, depending on the function values already obtained at the data sites chosen previously.  The algorithms sample more densely in places where it is required.  These
algorithms also automatically determine when to stop sampling the input function
and return their answers.



%%%%%%%%%%%%%%%%%%%%%%%%%%%%%%%%%%%%%%%%%%%%%%%%%%
\section{The Cone, $\cc$} \label{sec:cone}
%%%%%%%%%%%%%%%%%%%%%%%%%%%%%%%%%%%%%%%%%%%%%%%%%%

Linear splines will form the basis for our adaptive algorithms $A$ and $M$. The best linear spline error possible occurs for the Sobolev space of functions whose second derivatives have finite $L^{\infty}$-norm:
\[
\cw^{2,\infty}[a,b] : = \Bigl \{f \in C^1[a,b] : \norm{f''} : = \norm[{[a,b]}]{f''} : = \sup_{a \le x \le b} \abs{f''(x)} <  \infty \Bigr\}.
\]

To bound the error of the linear spline, our adaptive algorithms construct data-based upper bounds on $\norm[{[\alpha,
\beta]}]{f''}$ in terms of function values.  In \eqref{normbd} below, we derive an upper bound $B$, such that
\[
\norm[{[\alpha, \beta]}]{f''} \le B(f,[\alpha,\beta]) \quad \forall f \in \cc, \ a \le \alpha < \beta \le b,
\]
where $B$ depends only on values of $f$ sampled in the neighborhood of $[\alpha,
\beta]$. The set $\cc \subset \cw^{2,\infty}[a,b]$, for which this bound applies and for which our algorithms are guaranteed to succeed, consists of functions that are \emph{not too spiky}.  Our upper bound is homogeneous in the function, i.e.,
$B(cf,[\alpha,\beta]) = \abs{c} B(f,[\alpha,\beta])$ for all real $c$. Thus, $\cc$ is chosen to be a cone: $f \in \cc \implies cf \in \cc$ for all real
$c$. The precise definition of $\cc$ is given below in \eqref{conedef}. Cones of
functions are key to the theoretically justified adpative algorithms in
\cite{HicEtal14b}, \cite{Ton14a}, \cite{Din15a}, and \cite{HicRazYun15a}.

%--------------------------------------------------
\subsection{Newton Divided Differences} \label{sec:ndd}
%--------------------------------------------------

Functions in $\cw^{2,\infty}[a,b]$ may have $f''(x)$ undefined at countably many $x \in [a,b]$.  Thus, we define $f''(x)$ as an interval-valued function:
\begin{gather*}
f''(x) := \Bigl[\liminf_{t \to x} f''(t), \limsup_{t \to x} f''(t) \Bigr], \\
 f''(]\alpha, \beta[) := \bigcup_ {x \in ]\alpha,\beta[} f''(x), \qquad  ]\alpha,\beta[ \subset [a,b].
\end{gather*}
Furthermore, $\abs{f''(]\alpha, \beta[)}$ is the interval containing all values of $\abs{f''(x)}$ for $x \in ]\alpha,\beta[$.  We define a
measure of how small $\abs{f''(x)}$ can be for  $x \in [\alpha, \beta]$ as follows:
\begin{equation} \label{minfppdef}
m(f,\alpha, \beta) = \inf \abs{f''(]\alpha, \beta[)}.
\end{equation}
For any open intervals $I , J \subseteq [a,b]$, it follows from the definition~\eqref{minfppdef}
that
\begin{equation} \label{mdec}
m(f,I) \ge m(f,J) \qquad \forall I \subseteq J.
\end{equation}

We cannot determine $f''(x)$ for a specific $x$ based only on values of $f$. However,
we can use values of $f$ to provide an upper bound on $m(f,\alpha, \beta)$ via the Newton divided differences.

Let $p$ denote the Lagrange quadratic interpolating polynomial at the nodes
$\{x_1, x_2, x_3\}$, which may be written as
\begin{equation*}
p(x) : = f[x_1] + f[x_1, x_2](x-x_1) + f[x_1, x_2, x_3](x-x_1)(x-x_2),
\end{equation*}
where the $f[\cdots]$ are the Newton divided differences; see for example~\cite{CheKin12a}. In particular,
\begin{gather}
\nonumber
f[x_1] = f(x_1), \qquad f[x_1, x_2] = \frac{f[x_2] - f[x_1]}{x_2-x_1},  \\
f[x_1, x_2,x_3] = \frac{f[x_2,x_3] - f[x_1,x_2]}{x_3-x_1}. \label{divdiff}
\end{gather}
For any $f \in
\cw^{2,\infty}[a,b]$, the function $f - p$ has at least three distinct zeros on
$[x_1, x_3]$, so $f' - p'$ has at least two distinct zeros on this interval,
i.e., there exist $\xi_\pm$ with $x_1 < \xi_- < x_2 < \xi_+ < x_3$ with
$f'(\xi_\pm) - p'(\xi_{\pm}) = 0$. If $f''$ is continuous, then we can conclude
that $ f[x_1, x_2, x_3]= p''(\zeta) =f''(\zeta) $ for some $\zeta \in [x_1,
x_3]$. However, $\cw^{2,\infty}[a,b]$ contains functions without continuous
second derivatives. Fortunately, we can obtain a somewhat weaker---yet equally useful result---by the definition of \eqref{minfppdef},
\begin{multline*}
\bigabs{f[x_1, x_2, x_3]}(\xi_+  - \xi_-) = \abs{p'(\xi_+) - p'(\xi_-)} =  \abs{f'(\xi_+) - f'(\xi_-)} \\
= \abs{\int_{\xi_-}^{\xi_+} f''(x) \, \dif x} \ge m(f,x_1, x_3) (\xi_+  - \xi_-).
\end{multline*}
So the the data-based second-order divided difference provides an upper
bound on how small $\abs{f''}$ can be in the interval $[x_1, x_3]$:
\begin{equation}
\bigabs{f[x_1, x_2, x_3]} \ge m(f,x_1, x_3).
\end{equation}

%--------------------------------------------------
\subsection{The Cone Definition}  \label{sec:conedef}
%--------------------------------------------------

Let $\fh$ be some positive number, and let $\fC : [0,b-a] \to [1,\infty)$ be any
non-decreasing function. The cone of functions for which our algorithms will
apply only includes functions whose second derivatives do not change much in
magnitude over a short distance, namely,
\begin{multline} \label{conedef}
\nonumber \cc :=   \Bigl \{
 f  \in    \cw^{2,\infty}[a,b]:    \text{for all } x \in [a,b], \ x_\pm = x \pm h_\pm,
\\  h = \max(h_{\pm}) \in [0, \fh], \  \max\abs{f''(x)}  \le M(f,x,h_\pm) \Bigr \},
\end{multline}
where
\begin{multline*}
M(f,x,h_\pm)=\\
\begin{cases}
  \max\bigl(\fC(h_{-}) m(f,x_-,x),\fC(h_{+}) m(f,x,x_+)\bigr), & a \le x_- \le x_+ \le b,\\
\fC(h_{-}) m(f,x_-,x), & a \le x_- \le b <  x_+,\\
\fC(h_{+}) m(f,x,x_+), & x_- < a \le x_+ \le b.
\end{cases} %\quad
 %\Vastr
\end{multline*}
The parameter $\fh$ and the function $\fC$ define the cone $\cc$; increasing
one or both expands the cone to include more functions.

A function $f$ with $f''(\alpha) = f''(\beta) = \{0\} \ne f''((\alpha+\beta)/2)$ may
lie inside $\cc$ only if $\beta - \alpha > 2\fh$. Thus, $f''$ cannot have zeros two close to each other.  Except near the endpoints of
the interval, the definition of $\cc$ uses values of $f''$ on both sides of $x$
to bound $\max \abs{f''(x)}$. This allows $\cc$ to include functions with step
discontinuities in their second derivatives, provided that these discontinuities
do not occur too close to each other or too close to the ends of the interval.

To better understand $\cc$, we provide examples of functions lying outside $\cc$
and similar functions lying inside $\cc$. Consider these two functions defined
on $[-1,1]$ whose second derivatives oscillate wildly near $0$:
\begin{gather*}
f_1(x) = x^4 \sin(1/x), \\
 f_1''(x) = \begin{cases} (12x^2 - 1) \sin(1/x) -6 x \cos(1/x), & x \ne 0 \\
 [-1,1], & x = 0, \end{cases} \\
f_2(x) = 10  x^2 + f_1(x), \qquad f_2''(x) = 20+ f_1''(x).
\end{gather*}
These functions are plotted in Figure \ref{f1f2fig}. Because the $f''_1(x)$
takes on both signs for $x$ arbitrarily close to $0$ and on either side of $0$,
it follows that  $m(f_1,-h_-,0) = m(f_1,0,h_+) = 0$ for all $h_\pm \in [0,1]$.
However, $\max\abs{f''_1(0)} = 1$, so $f_1$ cannot lie inside
$\cc$ no matter how $\fh$ and $\fC$ are defined. On the other hand,
$m(f_2,\alpha, \beta) \ge 13.5$ for all $-1 \le \alpha < \beta \le 1$, and
$\max \abs{f''_2(x)} \le 27$ for all $x \in [-1,1]$, so $f_2 \in
\cc$ if $\fC(0) \ge 2$.

\begin{figure}[t]
\centering
\includegraphics[width=5.6cm]{f1f2plot.eps} \quad
\includegraphics[width=5.9cm]{f1closeplot.eps} \\
\includegraphics[width=5.6cm]{f1ppf2ppplot.eps} \quad
\includegraphics[width=5.9cm]{f2closeplot.eps}
\caption{The examples $f_1$ and $f_2$ and their second derivatives. Note that
$f_2''(x) = f_1''(x) + 20$.}
\label{f1f2fig}
\end{figure}

Consider the following function defined on $[-1,1]$, whose second derivative has jump discontinuities:
\begin{align*}
f_3(x) & = \begin{cases} \displaystyle
   \frac{1}{2\delta^2} \Bigl [4 \delta^2 + (x-c)^2 + (x-c-\delta)\abs{x-c-\delta}
\\ \qquad \qquad
    - (x-c+\delta)\abs{x-c+\delta} \Bigr ], & \abs{x-c} \le 2\delta,
\\ 0, & \text{otherwise},
\end{cases} \\
f''_3(x) & =
\begin{cases} \displaystyle
    \frac{1}{\delta^2} [1 + \sign(x-c-\delta) - \sign(x-c+\delta), & \abs{x-c} \le 2\delta,
\\ 0, & \text{otherwise}.
\end{cases}
\end{align*}
Here $c$ and $\delta$ are parameters. This function and its second derivative
are shown in Figure \ref{f3fig} for $-c=\delta = 0.2$. For this choice of
parameters, if $\fh < 0.1$, then for
any choice of $\fC : [0,b-a] \to [1,\infty)$ it follows that $f_3 \in \cc$.  We see this by examining two cases.  If $x \in [-0.6, 0.2]$, then for $h_\pm \in [0,\fh]$, and $x_\pm = x\pm h_\pm$, it follows that $-1 \le x_- \le x_+ \le 1$, and
\[
\max\bigl(m(f_3,x_-,x),m(f_3,x,x_+)\bigr) = \sup_{-1 \le x \le 1} \abs{f_3''(x)}  = 25.
\]
For $x \in [-1,1] \setminus [-0.6, 0.2]$ we note that $f_3''(x) = 0$. Thus, the definition of the cone is satisfied. 
However, if $\fh$ is too large, say, $0.1 < h_\pm \le \fh$, and $x_\pm = -0.5\pm
h_\pm$, then $m(f_3,x_-,-0.5)=m(f_3,-0.5,x_+)=0$. Since $f_3''(-0.5) = 25$,
now $f_3 \notin \cc$. This example illustrates how $\fh$ determines the width of
a spiky function that may still lie in $\cc$.

\begin{figure}[t]
\centering
\includegraphics[width=5.7cm]{f3plot.eps} \quad
\includegraphics[width=5.7cm]{f3ppplot.eps}
\caption{The example $f_3$ with $-c=\delta = 0.2$  and its piecwise constant second derivative.}
\label{f3fig}
\end{figure}

The above examples of functions outside $\cc$ have discontinuities in the second
derivative. Even if a function has sufficient smoothness, it \scnote{incomplete}

%--------------------------------------------------
\subsection{The Linear Spline and Its Error}
%--------------------------------------------------

Define a \emph{partition} of an interval $[a, b]$, denoted $\datasites$, to be
an ordered sequence of points that includes the endpoints of the interval,
$a=:x_0 < x_1 < \cdots < x_{n-1} < x_{n}:=b$, where $x_{i+1}-x_{i-2} <\fh, \forall i=2,\ldots, n-1$.  The linear spline
approximation to a function $f$ based on the partition $\datasites$ is denoted,
$S(f,\datasites)$ and defined as
\begin{multline} \label{splinedef}
S(f,\datasites)(x) =  \frac{x-x_i}{x_{i-1} - x_i} f(x_{i-1}) + \frac{x-x_{i-1}}{x_{i} - x_{i-1}}f(x_i), \\ x_{i-1} \le x \le x_i, \ i=1, \ldots, n.
\end{multline}
The error of this approximation is bounded as follows
\begin{equation} \label{appxerrbda}
\norm{f - S(f,\datasites)} \le \max_{i=1, \ldots, n} \frac{(x_i - x_{i-1})^2\norm[{[x_{i-1},x_i]}]{f''}}{8}.
\end{equation}
To construct an adaptive algorithm, one requires an upper bound on
$\norm[{[x_{i-1},x_i]}]{f''}$ in terms of function values. This can be done for
$f \in \cc$ using the correspondence between a second order difference and the
second derivative in~\eqref{minfppdef}.

For all $ x \in [x_{i-1},x_i]$,  denote
\begin{align*}
&h_- = x - x_{i-3}, \qquad x_- = x_{i-3},  \qquad i=3,4,\ldots,n,\\
 &h_+ = x_{i+2} - x, \qquad x_+ =  x_{i+2}, \qquad i=1,2,\ldots,n-2.
\end{align*}
Let
\begin{align*}
B_{+}(f,[x_{i-1},x_i])&=\begin{cases}
    \fC(x_{i+2}-x_{i-1})\abs{f[x_i,x_{i+1},x_{i+2}]},  & i=1,\ldots,n-2,
\\ 0, & i=n-1,n.
\end{cases}\\
 B_{-}(f,[x_{i-1},x_i])&=\begin{cases}
   0,  & i=1,2,
\\ \fC(x_{i}-x_{i-3})\abs{f[x_{i-3},x_{i-2},x_{i-1}])}, & i=3,\ldots,n.
\end{cases}
\end{align*}
At first, we consider the case of an interior interval, i.e., $i \in \{3, \ldots, n-2\}$. Then it follows by the definition of the cone that
\begin{equation*}
\abs{f''(x)} \le \max\bigl(\fC(h_-)m(f,x_{i-3},x),\fC(h_+)m(f,x,x_{i+2})\bigr)  \quad  \forall x \in [x_{i-1},x_i].
\end{equation*}
Applying the fact that $\fC$ is non-decreasing and property \eqref{mdec} yields
the following upper bound for the norm of $f''$ in terms of function values:
\begin{align}\label{normbd}
\nonumber
\MoveEqLeft{\norm[{[x_{i-1},x_i]}]{f''}}\\
\nonumber
 \le  & \sup_{x_{i-1} \le x \le x_i} \bigl[\max\bigl(\fC(x-x_{i-3})m(f,x_{i-3},x),\fC(x_{i+2}-x)m(f,x,x_{i+2})\bigr)\bigr]  \\
 %\nonumber & \qquad \qquad \times \max\bigl(m(f,x_{i-3},x),m(f,x,x_{i+2})\bigr] \\
\nonumber
 \le  &  \max\bigl(\fC(x_{i}-x_{i-3})m(f,x_{i-3},x_{i-1}),\fC(x_{i+2}-x_{i-1})m(f,x_i,x_{i+2})\bigr) \\
\nonumber  \le & \max\bigl(\fC(x_{i}-x_{i-3})\abs{f[x_{i-3},x_{i-2},x_{i-1}])},\bigr.  
 \fC(x_{i+2}-x_{i-1})\abs{f[x_i,x_{i+1},x_{i+2}]}\bigr) \\
 =  & \max\bigl(B_{\pm}(f,[x_{i-1},x_i])\bigr).
\end{align}

Similarly, we can obtain the same result for the sub-intervals on the left and right borders of the interval. Hence,
\eqref{normbd} is satisfied for all $i=1,\ldots,n$.
The bound in \eqref{normbd} combined with \eqref{appxerrbda} yield the
data-driven error bound for the linear spline:
\begin{equation} \label{appxerrbdb}
\norm{f - S(f,\datasites)} \le
\max_{i=1, \ldots, n} \frac{(x_i - x_{i-1})^2\max\bigl(B_{\pm}(f,[x_{i-1},x_i])\bigr)}{8} .
\end{equation}
The goal is to increase the number of nodes in the partition as needed to make
this error bound smaller than the desired tolerance.

%%%%%%%%%%%%%%%%%%%%%%%%%%%%%%%%%%%%%%%%%%%%%%%%%%
\section{The Function Approximation Algorithm, $A$}\label{sec:fappx}
%%%%%%%%%%%%%%%%%%%%%%%%%%%%%%%%%%%%%%%%%%%%%%%%%%

\begin{algoA}
For some finite interval, $[a,b]$, some fixed $\fh >0$, and some non-decreasing
$\fC:[0,b-a] \to [1, \infty)$, let $f:[a,b] \to \reals$ and $\abstol >0$ be
user inputs. Choose positive integers $n_{\lo}$ and $n_{\text{hi}}$ such that 
$n_{\text{hi}} \ge n_{\lo}$ and let  $n=n_{\ninit}-1$ where
\begin{equation}
\label{nodefinition}
n_{\ninit} = \max\left\{\left\lceil n_{\text{hi}}
\left(\frac{n_{\lo}}{n_{\text{hi}}}\right)^{\frac{1}{1+b-a}}\right\rceil ,5\right\}.
\end{equation}
Define the
following $n+1$ points and index set of intervals for which the error tolerance
is not yet satisified: $$x_i=a+\frac{i}{n}(b-a), \ i=0,\ldots,n, \qquad
\mathcal{I} = \{1,2,\ldots,n-1,n\},$$ and choose $\fC$. Compute $f[x_{i-1},
x_{i}, x_{i+1}]$ for $i = \{1,2,\ldots,n-1\}$ as in \eqref{divdiff}. Then do the
following:
\begin{enumerate}[\bf Step 1.]%\hspace{8.5ex}
%\renewcommand{\labelenumi}{\textbf{Step \arabic{enumi}.}}
\item \label{stage1} Check for convergence.
Let $\mathcal{I}=\mathcal{I}_+ \cup \mathcal{I}_-$, where
\[
\mathcal{I}_\pm = \left\{i \in \mathcal{I}: \frac{(x_i - x_{i-1})^2B_\pm(f,[x_{i-1},x_i])}{8}  > \abstol \right\}.
\]
If $\mathcal{I} = \emptyset$, return the linear spline $A(f,\varepsilon) = S(f, \{x_i\}_{i=0}^n)$ and terminate the algorithm.
Otherwise, continue to the next step.
\item \label{stage2}
Denote $\widehat{\mathcal{I}}=\widehat{\mathcal{I}}_{+1} \cup \widehat{\mathcal{I}}_{+2} \cup \widehat{\mathcal{I}}_{-1} \cup \widehat{\mathcal{I}}_{-2},$ where
\begin{multline*}
\widehat{\mathcal{I}}_{\pm k} = \left\{i \in \{1,2,\ldots,n\}: \frac{(x_j - x_{j-1})^2B_\pm(f,[x_{j-1},x_j])}{8}  > \abstol,\right.\\
 \left.j=i\mp k, 0<j\le n, k=1,2. \right\},
\end{multline*}
\[\widetilde{\mathcal{I}}=\left\{i \in \mathcal{I} \cup \widehat{\mathcal{I}}: x_i - x_{i-1}=\max\limits_{j \in \mathcal{I} \cup \widehat{\mathcal{I}} } x_j-x_{j-1} \right\}.\]
Split in half those intervals $[x_i,x_i-1]$ for which $i$ lies in $\widetilde{\mathcal{I}}$.
Update $n$, $\{x_i\}_{i=0}^n$, $\mathcal{I}$, and the $f[x_{i-1}, x_{i}, x_{i+1}]$ accordingly.  Return to Step~\ref{stage1}.
\end{enumerate}
\end{algoA}

\subsection{Computational Cost}

In this part, we investigate the computational cost of our locally adaptive algorithm. Let $n_0$ denote the initial number of intervals. We firstly introduce one new notation $\Ixl$. Let
$\ell(x)$ be defined such that
$I_{x,\ell(x)}$ is the final interval which contains $x$ in the final discretization where width of $I_{x,\ell(x)}$ is
$\frac{2^{-\ell(x)}(b-a)}{n_0}$.
Thus the computational cost should be
\[
\cost(S,f,\varepsilon) =\int_a^b \frac{2^{\ell(x)}n_0}{(b-a} dx +1 = \frac{n_0}{(b-a} \int_a^b 2^{\ell(x)} dx +1
\]
Thus, for subinterval $I_{x,l}$, we have the error bound from~\eqref{appxerrbdb}
$$\errest(x,l)=\frac{(x_i - x_{i-1})^2B(f,[x_{i-1},x_i])}{8}.$$
Let $j=\lfloor\frac{n_0 2^l x}{b-a}\rfloor$. Denote
\[\bar{I}_{x,l}=\left[\max\left(a, \frac{(j-2)2^{-l}(b-a)}{n_0}\right), \min\left(b, \frac{(j+3)2^{-l}(b-a)}{n_0}\right)\right]
\].

For $i \in \{3, \ldots, L-2\}$,  as $\fC$ a is non-decreasing function and by \eqref{normbd}, we know
\begin{align*}
B(f,[x_{i-1},x_i]) & \le \fC (2(x_i-x_{i-1}))\max\limits_{x_{i-3}\le t \le x_{i+2}}\abs{f''(t)}\\
& \le \fC\left(\frac{2 \cdot2^{-l}(b-a)}{n_0}\right) \norm[\bar{I}_{x,l}]{f''}.
%&\le \fC\left(\frac{2(b-a)}{n_0}\right) \max\limits_{\max\{a,x-\delta\} \le t \le \min\{x+\delta, b\}}\abs{f''(t)},\\
\end{align*}
Similar for $i\in \{1,2\}$, we have
\begin{align*}
B(f,[x_{i-1},x_i]) & \le \fC(3(x_i-x_{i-1}))\max\limits_{x_{i}\le t \le x_{i+2}}\abs{f''(t)}\\
& \le \fC\left(\frac{3\cdot2^{-l}(b-a)}{n_0}\right) \norm[\bar{I}_{x,l}]{f''}.
%&\le \fC\left(\frac{2(b-a)}{n_0}\right) \max\limits_{\max\{a,x-\delta\} \le t \le \min\{x+\delta, b\}}\abs{f''(t)},\\
\end{align*}
Thus for $i \in \{L-1,L\}$, we have
\begin{align*}
B(f,[x_{i-1},x_i]) & \le \fC(3(x_i-x_{i-1})) \max\limits_{x_{i-3}\le t \le x_{i-1}}\abs{f''(t)}\\
& \le \fC\left(\frac{3\cdot2^{-l}(b-a)}{n_0}\right) \norm[\bar{I}_{x,l}]{f''}.
%&\le \fC\left(\frac{2(b-a)}{n_0}\right) \max\limits_{\max\{a,x-\delta\} \le t \le \min\{x+\delta, b\}}\abs{f''(t)},\\
\end{align*}
Therefore, for any $i \in \{1,2,\ldots, L\}$ we can obtain
\begin{align*}
B(f,[x_{i-1},x_i]) & \le \fC\left(\frac{3\cdot2^{-l}(b-a)}{n_0}\right)  \norm[\bar{I}_{x,l}]{f''}.
\end{align*}

Thus
we can obtain an upper bound on $\errest(x,l)$ in terms of $ \norm[\bar{I}_{x,l}]{f''}$:
\[
 \errest(x,l)  \le C_l \norm[\bar{I}_{x,l}]{f''}, \qquad \forall x \in [a,b], \ l \in \natzero,
\]
where
\[\qquad C_l=\fC\left(\frac{3\cdot2^{-l}(b-a)}{n_0}\right) \frac{2^{-2l}(b-a)^2}{8n^2_0}
\]
depends on $n_0$, $\fC$, and $l$, but not on $f$.

We define
\begin{equation}\label{eqn:defoflx}
L(x) = \min \left\{ l \in \natzero : C_l  \norm[\bar{I}_{x,l}]{f''} \le \varepsilon\right\} .
\end{equation}

We want to show that $\forall x$, if the interval containing $x$ becomes as the same as
$I_{x,L(x)}$ that it cannot be cut further, i.e., $\ell(x) \le L(x)$. 

Suppose $I_{x,L(x)}$ has to be cut,i.e. $\ell(x)>L(x)$, where $I_{x,L(x)}=[x_{i-1},x_i]$ for some $i$. Only in five different cases, interval $[x_{i-1},x]$ needs to be cut.
We will show none of them satisfied.
\begin{enumerate}
  \item $[x_{i-1},x]$ doesn't satisfy the error tolerance, i.e.,
  $$\frac{B(f,[x_{i-1},x_i])}{8}(x_i-x_{i-1})^2 > \abstol.$$
  However, by the definition of $L(x)$, it is impossible.
  \item $[x_{i-2},x_{i-1}]$ doesn't satisfy the error tolerance.  By the algorithm, we know
  \begin{align*}&\frac{B_+(f,[x_{i-2},x_{i-1}])}{8}(x_{i-1}-x_{i-2})^2 > \abstol\\
  &x_i-x_{i-1} \ge  x_{i+1}-x_{i} \qquad x_i-x_{i-1} \ge x_{i-1}-x_{i-2}.
  \end{align*}
  But from the definition, we know
  \begin{align*}
  & \frac{B_+(f,[x_{i-2},x_{i-1}])}{8}(x_{i-1}-x_{i-2})^2 \\ \le &\frac{\fC(x_{i+1}-x_{i-2})f[x_{i-1},x_{i},x_{i+1}]}{8}(x_{i-1}-x_{i-2})^2\\
    \le & \fC\left(\frac{3\cdot2^{-l}(b-a)}{n_0}\right) \frac{2^{-2l}(b-a)^2}{8n^2_0}\norm[{[x_{i-1},x_{i+1}]}]{f''} \\
     \le & C_{L(x)} \norm[\bar{I}_{x,L(x)}]{f''} < \abstol
  \end{align*}
  Contradiction.
  \item $[x_{i-3},x_{i-2}]$ doesn't satisfy the error tolerance. By the algorithm, we know
  \begin{align*}
  & \frac{B_+(f,[x_{i-3},x_{i-2}])}{8}(x_{i-2}-x_{i-3})^2 > \abstol \\&x_i-x_{i-1} \ge x_{i-2}-x_{i-3} \qquad x_i-x_{i-1} \ge x_{i-1}-x_{i-2}.
  \end{align*}
  But from the definition, we know
  \begin{align*}
  & \frac{B_+(f,[x_{i-3},x_{i-2}])}{8}(x_{i-2}-x_{i-3})^2 \\  \le &\frac{\fC(x_i-x_{i-3})f[x_{i-2},x_{i-1},x_{i}]}{8}(x_{i-2}-x_{i-3})^2\\
    \le & \fC\left(\frac{3\cdot2^{-l}(b-a)}{n_0}\right) \frac{2^{-2l}(b-a)^2}{8n^2_0}\norm[{[x_{i-2},x_{i}]}]{f''} \\
     \le & C_{L(x)} \norm[\bar{I}_{x,L(x)}]{f''}< \abstol.
  \end{align*}
  Contradiction.
  \item $[x_{i},x_{i+1}]$ doesn't satisfy the error tolerance. By the algorithm, we know
  \begin{align*}& \frac{B_-(f,[x_{i},x_{i+1}])}{8}(x_{i+1}-x_{i})^2 > \abstol \\ & x_i-x_{i-1} \ge x_{i+1}-x_{i} \qquad x_i-x_{i-1} \ge x_{i-1}-x_{i-2}.\end{align*}
  But from the definition, we know
  \begin{align*}
  & \frac{B_-(f,[x_{i},x_{i+1}])}{8}(x_{i+1}-x_{i})^2 \\ \le &\frac{\fC(x_{i+1}-x_{i-2})f[x_{i-2},x_{i-1},x_{i}]}{8}(x_{i+1}-x_{i})^2\\
    \le & \fC\left(\frac{3\cdot2^{-l}(b-a)}{n_0}\right) \frac{2^{-2l}(b-a)^2}{8n^2_0}\norm[{[x_{i-2},x_{i}]}]{f''} \\
     \le & C_{L(x)} \norm[\bar{I}_{x,L(x)}]{f''}< \abstol.
  \end{align*}
  Contradiction.
  \item $[x_{i+1},x_{i+2}]$ doesn't satisfy the error tolerance. By the algorithm, we know
  \begin{align*} &\frac{B_-(f,[x_{i+1},x_{i+2}])}{8}(x_{i+2}-x_{i+1})^2 > \abstol \\
  & x_i-x_{i-1} \ge x_{i+1}-x_{i} \qquad x_i-x_{i-1} \ge x_{i+2}-x_{i+1}.
  \end{align*}
  But from the definition, we know
  \begin{align*}
  & \frac{B_-(f,[x_{i+1},x_{i+2}])}{8}(x_{i+2}-x_{i+1})^2 \\ \le &\frac{\fC(x_{i+2}-x_{i-1})f[x_{i-1},x_{i},x_{i+1}]}{8}(x_{i+2}-x_{i+1})^2\\
    \le & \fC\left(\frac{3\cdot2^{-l}(b-a)}{n_0}\right) \frac{2^{-2l}(b-a)^2}{8n^2_0}\norm[{[x_{i-1},x_{i+1}]}]{f''} \\
     \le & C_{L(x)} \norm[\bar{I}_{x,L(x)}]{f''}<\abstol.
  \end{align*}
  Contradiction.
\end{enumerate}

Hence, we prove $\ell(x) \le L(x)$. Next we want to investigate $2^{L(x)}$.
$$
2^{L(x)}=2^{\min \left\{ l \in \natzero : C_l  \norm[\bar{I}_{x,l}]{f''} \le \varepsilon\right\}} = \min\left\{2^l: l \in  \natzero, C_l \norm[\bar{I}_{x,l}]{f''} \le \varepsilon\right\}.$$
Denote
\begin{align}\label{mxdef}
M(x) & = \min\left\{2^l:  \fC\left(\frac{3(b-a)}{n_0}\right) \frac{2^{-2l}(b-a)^2}{8n^2_0} \norm[\bar{I}_{x,l}]{f''}  \le \varepsilon,  l \in  \natzero\right\}\nonumber \\
 & = \min\left\{2^l:  2^l \ge \sqrt{\frac{\fC\left(\frac{3 \cdot 2^{-l}(b-a)}{n_0}\right)  (b-a)^2  \norm[\bar{I}_{x,l}]{f''} }{8n^2_0\varepsilon}},  l \in  \natzero\right\}.
\end{align}
We obtain the upper bound on computational cost of Algorithm \texttt{funappx\_g}.

\begin{theorem}\label{thm:cost}
Let $A(f,\varepsilon)$ be the adaptive linear spline defined by Algorithm \textnormal{\texttt{funappx\_g}}, and let $n_0$, and $\varepsilon$ be the inputs and parameters described there. Let $\cc$ be the cone of functions defined in~\eqref{conedef}.
Let $M(x)$ is defined in~\eqref{mxdef}.
Then it follows that Algorithm \textnormal{\texttt{funappx\_g}} is successful for all functions in $\cc$,  i.e.,  $\norm[\infty]{f - S(f, \cdot)} \le \varepsilon$.  Moreover, the cost of this algorithm is bounded above as follows:
$$\cost(S,f,\varepsilon) \le \frac{n_0}{b-a}\int_a^b M(x) dx +1.$$
\end{theorem}

\begin{proof}
For subinterval $I_{x,\ell(x)}$, where $\ell(x)$ is defined as above,
which means Algorithm \texttt{funappx\_g} satisfied error tolerance on $I_{x,\ell(x)}$ with two points.
If we take out the right end point of the subinterval, then the density of the cost on $I_{x,\ell(x)}$ can be considered as
$$\frac{n_0}{2^{-\ell(x)}(b-a)}.$$
Thus, we obtain the cost on whole interval $[a,b]$ should be
$$\cost(S,f,\varepsilon)  = \int_a^b \frac{n_0}{2^{-\ell(x)}(b-a)}dx +1= \frac{n_0}{(b-a)}\int_a^b 2^{\ell(x)}dx +1.
$$
If $L(x)$ is defined in~\eqref{eqn:defoflx}, then we know $\ell(x) \le L(x)$. Therefore we obtain an upper bound
$$\cost(S,f,\varepsilon)  \le \frac{n_0}{(b-a)}\int_a^b 2^{L(x)}dx +1.
$$
As $M(x)=2^{L(x)}$ by the definition~\eqref{mxdef}, we obtain
$$\cost(S,f,\varepsilon)  \le \frac{n_0}{(b-a)}\int_a^b M(x) dx +1.
$$
\end{proof}

From Theorem~\ref{thm:cost}, we know for very small $\varepsilon$,
$l$ tends to $\infty$, $ \norm[\bar{I}_{x,l}]{f''} $ tends to $|f''(x)|$.
Thus we can have
$$ M(x) \lesssim 2\sqrt{\frac{\fC\left(\frac{3 \cdot 2^{-l}(b-a)}{n_0}\right)  (b-a)^2 |f''(x)|}{8n^2_0\varepsilon}}.$$
The upper bound on computational cost tends to
\begin{align*}
\cost(S,f,\varepsilon)  & \lesssim \frac{n_0}{(b-a)}\int_a^b 2\sqrt{\frac{\fC\left(\frac{3 \cdot 2^{-l}(b-a)}{n_0}\right)  (b-a)^2 |f''(x)|}{8n^2_0\varepsilon}} dx +1 \\
& =\sqrt{\frac{\fC\left(\frac{3 \cdot 2^{-l}(b-a)}{n_0}\right)}{2\varepsilon}}\int_a^b \sqrt{|f''(x)|} dx +1\\
& =\sqrt{\frac{\fC\left(\frac{3 \cdot 2^{-l}(b-a)}{n_0}\right)}{2\varepsilon}}\norm[1]{\sqrt{|f''|}} dx +1,
\end{align*}
where $\norm[1]{g}=\int_a^b |g''(x)| dx$.
However, when $\varepsilon$ is not that small, the computational cost does not only depend on $\int_a^b|f''(x)|dx$ but also depends on where and how the peaky parts are located.\\


we need a theorem that works.\\
Numerical examples.\\
We need computational cost\\
complexity bound 

\section{The Minimization Algorithm, $M$}

we need a theorem that works.\\
Numerical examples.\\
We need computational cost\\
complexity bound

\section{Discussion}

%%%%%%%%%%%%%%%%%%%%%%%%%%%%%%%%%%%%%%%%%%%%%%%%%%
\section*{References}
%%%%%%%%%%%%%%%%%%%%%%%%%%%%%%%%%%%%%%%%%%%%%%%%%%

\bibliography{FJH23,FJHOwn23}

\end{document}








