\subsection{Computational Cost}

In this part, we investigate the computational cost of our locally adaptive algorithm. Let $n_0$ denote the initial number of intervals, and
 $h_0=(b-a)/{n_0}$ be the width of the subinterval. We firstly introduce one new notation $\Ixl$, which is the unique interval contains $x$ with
 the width $2^{-l}h_0$.
\[\Ixl=\left[a+(j-1)2^{-l}h_0,a+j \ 2^{-l}h_0\right), j=\left\lceil\frac{(x-a)2^l}{h_0}\right\rceil, \ l = \mathbb{N}_0.\]
Let
$\ell(x)$ be defined such that
$I_{x,\ell(x)}$ is the final interval which contains $x$ in the final discretization.
Let
\[\bar{I}_{x,l}=\left[a+\max(0,j-3)2^{-l}h_0, a+ \min(j+2,2^ln_0)2^{-l}h_0\right],
\] with the same $j$ as above. This is an interval with generally five times the length of $\Ixl$ that also contains $\Ixl$.

Let
\begin{equation}\label{eqn:defoflx}
L(x) = \min \left\{ l \in \natzero : C_l  \norm[\bar{I}_{x,l}]{f''} \le \abstol \right\},
\end{equation}
where
\[\qquad C_l=\fC\left(3\cdot2^{-l}h_0\right)\frac{(2^{-l}h_0)^2}{8}
\]
depends on $n_0$, $\fC$, and $l$, but not on $f$.

We want to show that $\ell(x) \le L(x)$. If not, the interval $I_{x,L(x)}$ must be split. Let the closure of $\mathcal{I}_{x,L(x)}$ be
denoted $[x_{i-1},x_i]$, where $x_i-x_{i-1}=2^{-L(x)}h_0$.
By the algorithm, it is necessary that at least one of the following two conditions must hold:
\[i \in \mathcal{I} \cap \widetilde{\mathcal{I}}
\text{ or } \widehat{\mathcal{I}}\cap \widetilde{\mathcal{I}}. \]
Let us investigate these two conditions separately. 
\begin{enumerate}
  \item $i \in \mathcal{I} \cap \widetilde{\mathcal{I}}$\\
  There are two different cases in this situation. We firstly investigate $i \in \mathcal{I}_+ \cap \widetilde{\mathcal{I}}$.
  Thus 
  \begin{equation}
  \label{conditionone} x_{i}-x_{i-1} \ge x_{j}-x_{j-1}, \qquad   j=i+1,i+2, \qquad [x_i,x_{i+2}] \subseteq \bar{I}_{x,l}. 
  \end{equation}
  Then
  \begin{align*}
  \abstol < & \frac{B_+(f,[x_{i-1},x_i])}{8}(x_i-x_{i-1})^2 \\
   \le &\frac{\fC(x_{i+2}-x_{i-1})f[x_{i},x_{i+1},x_{i+2}]}{8}(x_{i}-x_{i-1})^2 \qquad \text{by } \eqref{bpf} \\
  %\text{since } x_{i}-x_{i-1} \ge x_{j}-x_{j-1}, j=i+1,i+2 \ \ \
  \le & \fC\left(3\cdot2^{-L(x)}h_0\right) \frac{2^{-2L(x)}h_0^2}{8}\norm[{[x_{i},x_{i+2}]}]{f''} \qquad \text{by } \eqref{conditionone}  \\
     \le & C_{L(x)} \norm[\bar{I}_{x,L(x)}]{f''} \le \abstol.
  \end{align*}
  We can also use the similar argument to obtain
   \begin{align*}
  \abstol < \frac{B_-(f,[x_{i-1},x_i])}{8}(x_i-x_{i-1})^2  \le
      C_{L(x)} \norm[\bar{I}_{x,L(x)}]{f''} \le \abstol.
  \end{align*}
  In both cases, there is a contradiction. So we cannot have $i \in \mathcal{I} \cap \widetilde{\mathcal{I}}$.
  \item $i \in \widehat{\mathcal{I}}\cap \widetilde{\mathcal{I}}$\\
  There are four different cases in this situation. We first consider $i \in \widehat{\mathcal{I}}_{+1} \cap \widetilde{\mathcal{I}}$.
  Thus
  \begin{equation} \label{conditiontwo}
  x_{i}-x_{i-1} \ge x_{j}-x_{j-1},\qquad j=i-1,i+1, \qquad [x_{i-1},x_{i+1}] \subseteq \bar{I}_{x,l}.
  \end{equation}
  Then
  \begin{align*}
  \abstol \le & \frac{B_+(f,[x_{i-2},x_{i-1}])}{8}(x_{i-1}-x_{i-2})^2 \\ 
 \le  &\frac{\fC(x_{i+1}-x_{i-2})f[x_{i-1},x_{i},x_{i+1}]}{8}(x_{i-1}-x_{i-2})^2 \qquad  \text{by } \eqref{bpf}   \\
 \le & \fC\left(3\cdot2^{-L(x)}h_0\right) \frac{2^{-2L(x)}h_0^2}{8}\norm[{[x_{i-1},x_{i+1}]}]{f''}  \qquad \text{by } \eqref{conditiontwo}    \\
     \le & C_{L(x)} \norm[\bar{I}_{x,L(x)}]{f''} < \abstol.
  \end{align*}
  We can also get contradiction for other three cases. So we cannot have $i \in \widehat{\mathcal{I}}\cap \widetilde{\mathcal{I}}$.
\end{enumerate}
Hence, we prove $\ell(x) \le L(x)$. Next we want to investigate $2^{L(x)}$.
\begin{align*}
2^{L(x)}&=2^{\min \left\{ l \in \natzero : C_l  \norm[\bar{I}_{x,l}]{f''} \le \varepsilon\right\}} = \min\left\{2^l: l \in  \natzero, C_l \norm[\bar{I}_{x,l}]{f''} \le \varepsilon\right\}\\
 & = \min\left\{2^l:  \fC\left(3\cdot 2^{-l} h_0\right) \frac{2^{-2l}(h_0^2}{8} \norm[\bar{I}_{x,l}]{f''}  \le \varepsilon,  l \in  \natzero\right\}\nonumber \\
 & = \min\left\{2^l:  2^l \ge \sqrt{\frac{\fC\left(3\cdot 2^{-l} h_0\right) h_0^2 \norm[\bar{I}_{x,l}]{f''} }{8\varepsilon}},  l \in  \natzero\right\}.
\end{align*}
We obtain the upper bound on computational cost of Algorithm \texttt{funappx\_g}.

\begin{theorem}\label{thm:cost}
Let $A(f,\varepsilon)$ be the adaptive linear spline defined by Algorithm \textnormal{\texttt{funappx\_g}}, and let $n_0$, and $\varepsilon$ be the inputs and parameters described there. Let $\cc$ be the cone of functions defined in~\eqref{conedef}.
Let $L(x)$ is defined in~\eqref{eqn:defoflx}.
Then it follows that Algorithm \textnormal{\texttt{funappx\_g}} is successful for all functions in $\cc$,  i.e.,  $\norm[\infty]{f - S(f, \cdot)} \le \varepsilon$.  Moreover, the cost of this algorithm is bounded above as follows:
$$\cost(S,f,\varepsilon) \le \frac{n_0}{b-a}\int_a^b 2^{L(x)} dx +1.$$
\end{theorem}

\begin{proof}
For subinterval $I_{x,\ell(x)}$, where $\ell(x)$ is defined as above,
which means Algorithm \texttt{funappx\_g} satisfied error tolerance on $I_{x,\ell(x)}$ with two points.
If we take out the right end point of the subinterval, then the density of the cost on $I_{x,\ell(x)}$ can be considered as
$$\frac{n_0}{2^{-\ell(x)}(b-a)}.$$
Thus, we obtain the cost on whole interval $[a,b]$ should be
$$\cost(S,f,\varepsilon)  = \int_a^b \frac{n_0}{2^{-\ell(x)}(b-a)}dx +1= \frac{n_0}{(b-a)}\int_a^b 2^{\ell(x)}dx +1.
$$
If $L(x)$ is defined in~\eqref{eqn:defoflx}, then we know $\ell(x) \le L(x)$. Therefore we obtain an upper bound
$$\cost(S,f,\varepsilon)  \le \frac{n_0}{(b-a)}\int_a^b 2^{L(x)}dx +1.
$$
\end{proof}

From Theorem~\ref{thm:cost}, we know for very small $\varepsilon$,
$L(x)$ tends to $\infty$, $ \norm[\bar{I}_{x,L(x)}]{f''} $ tends to $|f''(x)|$ and $3\cdot 2^{-L(x)}h_0$ tends to 0.
Thus we can have
$$ 2^{L(x)} \approx \sqrt{\frac{\fC\left(0\right)  h_0^2 |f''(x)|}{8\varepsilon}}.$$
The upper bound on computational cost tends to
\begin{align*}
\cost(S,f,\varepsilon)  & \lesssim \frac{n_0}{(b-a)}\int_a^b \sqrt{\frac{\fC\left(0\right)  h_0^2 |f''(x)|}{8\varepsilon}} dx +1 \\
& =\sqrt{\frac{\fC\left(0\right)}{8\varepsilon}}\int_a^b \sqrt{|f''(x)|} dx +1\\
& =\sqrt{\frac{\fC\left(0\right)}{8\varepsilon}}\norm[1]{\sqrt{|f''|}} dx +1,
\end{align*}
where $\norm[1]{g}=\int_a^b |g''(x)| dx$.
However, when $\varepsilon$ is not that small, the computational cost does not only depend on $\int_a^b|f''(x)|dx$ but also depends on where and how the peaky parts are located.\\
